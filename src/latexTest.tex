% Options for packages loaded elsewhere
\PassOptionsToPackage{unicode}{hyperref}
\PassOptionsToPackage{hyphens}{url}
%
\documentclass[
]{article}
\usepackage{amsmath,amssymb}
\usepackage{lmodern}
\usepackage{iftex}
\ifPDFTeX
  \usepackage[T1]{fontenc}
  \usepackage[utf8]{inputenc}
  \usepackage{textcomp} % provide euro and other symbols
\else % if luatex or xetex
  \usepackage{unicode-math}
  \defaultfontfeatures{Scale=MatchLowercase}
  \defaultfontfeatures[\rmfamily]{Ligatures=TeX,Scale=1}
\fi
% Use upquote if available, for straight quotes in verbatim environments
\IfFileExists{upquote.sty}{\usepackage{upquote}}{}
\IfFileExists{microtype.sty}{% use microtype if available
  \usepackage[]{microtype}
  \UseMicrotypeSet[protrusion]{basicmath} % disable protrusion for tt fonts
}{}
\makeatletter
\@ifundefined{KOMAClassName}{% if non-KOMA class
  \IfFileExists{parskip.sty}{%
    \usepackage{parskip}
  }{% else
    \setlength{\parindent}{0pt}
    \setlength{\parskip}{6pt plus 2pt minus 1pt}}
}{% if KOMA class
  \KOMAoptions{parskip=half}}
\makeatother
\usepackage{xcolor}
\usepackage{color}
\usepackage{fancyvrb}
\newcommand{\VerbBar}{|}
\newcommand{\VERB}{\Verb[commandchars=\\\{\}]}
\DefineVerbatimEnvironment{Highlighting}{Verbatim}{commandchars=\\\{\}}
% Add ',fontsize=\small' for more characters per line
\newenvironment{Shaded}{}{}
\newcommand{\AlertTok}[1]{\textcolor[rgb]{1.00,0.00,0.00}{\textbf{#1}}}
\newcommand{\AnnotationTok}[1]{\textcolor[rgb]{0.38,0.63,0.69}{\textbf{\textit{#1}}}}
\newcommand{\AttributeTok}[1]{\textcolor[rgb]{0.49,0.56,0.16}{#1}}
\newcommand{\BaseNTok}[1]{\textcolor[rgb]{0.25,0.63,0.44}{#1}}
\newcommand{\BuiltInTok}[1]{#1}
\newcommand{\CharTok}[1]{\textcolor[rgb]{0.25,0.44,0.63}{#1}}
\newcommand{\CommentTok}[1]{\textcolor[rgb]{0.38,0.63,0.69}{\textit{#1}}}
\newcommand{\CommentVarTok}[1]{\textcolor[rgb]{0.38,0.63,0.69}{\textbf{\textit{#1}}}}
\newcommand{\ConstantTok}[1]{\textcolor[rgb]{0.53,0.00,0.00}{#1}}
\newcommand{\ControlFlowTok}[1]{\textcolor[rgb]{0.00,0.44,0.13}{\textbf{#1}}}
\newcommand{\DataTypeTok}[1]{\textcolor[rgb]{0.56,0.13,0.00}{#1}}
\newcommand{\DecValTok}[1]{\textcolor[rgb]{0.25,0.63,0.44}{#1}}
\newcommand{\DocumentationTok}[1]{\textcolor[rgb]{0.73,0.13,0.13}{\textit{#1}}}
\newcommand{\ErrorTok}[1]{\textcolor[rgb]{1.00,0.00,0.00}{\textbf{#1}}}
\newcommand{\ExtensionTok}[1]{#1}
\newcommand{\FloatTok}[1]{\textcolor[rgb]{0.25,0.63,0.44}{#1}}
\newcommand{\FunctionTok}[1]{\textcolor[rgb]{0.02,0.16,0.49}{#1}}
\newcommand{\ImportTok}[1]{#1}
\newcommand{\InformationTok}[1]{\textcolor[rgb]{0.38,0.63,0.69}{\textbf{\textit{#1}}}}
\newcommand{\KeywordTok}[1]{\textcolor[rgb]{0.00,0.44,0.13}{\textbf{#1}}}
\newcommand{\NormalTok}[1]{#1}
\newcommand{\OperatorTok}[1]{\textcolor[rgb]{0.40,0.40,0.40}{#1}}
\newcommand{\OtherTok}[1]{\textcolor[rgb]{0.00,0.44,0.13}{#1}}
\newcommand{\PreprocessorTok}[1]{\textcolor[rgb]{0.74,0.48,0.00}{#1}}
\newcommand{\RegionMarkerTok}[1]{#1}
\newcommand{\SpecialCharTok}[1]{\textcolor[rgb]{0.25,0.44,0.63}{#1}}
\newcommand{\SpecialStringTok}[1]{\textcolor[rgb]{0.73,0.40,0.53}{#1}}
\newcommand{\StringTok}[1]{\textcolor[rgb]{0.25,0.44,0.63}{#1}}
\newcommand{\VariableTok}[1]{\textcolor[rgb]{0.10,0.09,0.49}{#1}}
\newcommand{\VerbatimStringTok}[1]{\textcolor[rgb]{0.25,0.44,0.63}{#1}}
\newcommand{\WarningTok}[1]{\textcolor[rgb]{0.38,0.63,0.69}{\textbf{\textit{#1}}}}
\setlength{\emergencystretch}{3em} % prevent overfull lines
\providecommand{\tightlist}{%
  \setlength{\itemsep}{0pt}\setlength{\parskip}{0pt}}
\setcounter{secnumdepth}{-\maxdimen} % remove section numbering
\ifLuaTeX
  \usepackage{selnolig}  % disable illegal ligatures
\fi
\IfFileExists{bookmark.sty}{\usepackage{bookmark}}{\usepackage{hyperref}}
\IfFileExists{xurl.sty}{\usepackage{xurl}}{} % add URL line breaks if available
\urlstyle{same} % disable monospaced font for URLs
\hypersetup{
  hidelinks,
  pdfcreator={LaTeX via pandoc}}

\author{}
\date{}

\begin{document}

\hypertarget{header-n0}{%
\section{一般带状线性方程组的分裂法}\label{header-n0}}

\tableofcontents

\hypertarget{header-n3}{%
\subsection{描述:}\label{header-n3}}

考虑带状线性方程组\(Ax=r\),其中\(A\)是上、下半带宽分别为\(\beta、\alpha\)的带状矩阵,现将其分块为

\begin{bmatrix}
 A^{(0,0)} & A^{(0,1)} &           &           &           \\
  A^{(1,0)} & A^{(1,1)} & A^{(1,2)} &           &           \\
            & A^{(2,1)} & A^{(2,2)} &  \ddots   &           \\
            &           &  \ddots   &  \ddots   & A^{(3,4)} \\
            &           &           & A^{(4,3)} & A^{(4,4)}
\end{bmatrix}
\begin{bmatrix}
        x^{(0)}         \\
        x^{(1)}         \\
        x^{(2)}         \\
        \vdots          \\
        x^{(p-1)}       \\
\end{bmatrix}
=
\begin{bmatrix}
        r^{(0)}     \\
    r^{(1)}     \\
    r^{(2)}     \\
    \vdots      \\
    r^{(p-1)}   \\
\end{bmatrix}

\(\alpha\)包含主对角线的,\(\beta\)不包含主对角线。

其中\(A^{(i,j)}\)是\(n_i \times n_j\)矩阵,\(x^{(i)}=[x_{m_i+1},\cdots,x_{m_i+ni}]^T\)与\(r^{(i)}=[r_{m_i+1},r_{m_i+2},\cdots,r_{m_i+n_i}]^T\)为\(n_i\)维向量,\(\alpha+beta-2 \le \min\{n_i:0\le i \le p-1\}\),且

\[\left\{\begin{matrix}
 m_0=0\\
 m_{i+1}=m_i+n_i,i=1,2,\cdots,p
\end{matrix}\right.\]

当\(n=24,\alpha=2,\beta=2,p=4\)且\(n_i \equiv 6\)时,矩阵\(A\)的分裂方式如下图所示.

\[\small
\begin{matrix}
 b_1 & c_1 & d_1 &  &  &  &  &  &  &  &  &  &  &  &  &  &  &  &  &  &  &  &  & \\
 a_2 & b_2 & c_2  & d_2 &  &  &  &  &  &  &  &  &  &  &  &  &  &  &  &  &  &  &  & \\
  &  a_3 & b_3 & c_3 & d_3 &  &  &  &  &  &  &  &  &  &  &  &  &  &  &  &  &  &  & \\
  &  &  a_4 & b_4 & c_4 & d_4 &  &  &  &  &  &  &  &  &  &  &  &  &  &  &  &  &  & \\
  &  &  &  a_5 & b_5 & c_5 & d_5 &  &  &  &  &  &  &  &  &  &  &  &  &  &  &  &  & \\
  &  &  &  &  a_6 & b_6 & c_6 & d_6 &  &  &  &  &  &  &  &  &  &  &  &  &  &  &  & \\ \hline
  &  &  &  &  &  a_7 & b_7 & c_7 & d_7 &  &  &  &  &  &  &  &  &  &  &  &  &  &  & \\
  &  &  &  &  &  &  a_8 & b_8 & c_8 & d_8 &  &  &  &  &  &  &  &  &  &  &  &  &  & \\
  &  &  &  &  &  &  &  a_9 & b_9 & c_9 & d_9 &  &  &  &  &  &  &  &  &  &  &  &  & \\
  &  &  &  &  &  &  &  &  a_{10} & b_{10} & c_{10} & d_{10} &  &  &  &  &  &  &  &  &  &  &  & \\
  &  &  &  &  &  &  &  &  &  a_{11} & b_{11} & c_{11} & d_{11} &  &  &  &  &  &  &  &  &  &  & \\
  &  &  &  &  &  &  &  &  &  &  a_{12} & b_{12} & c_{12} & d_{12} &  &  &  &  &  &  &  &  &  & \\ \hline
  &  &  &  &  &  &  &  &  &  &  &  a_{13} & b_{13} & c_{13} & d_{13} &  &  &  &  &  &  &  &  & \\
  &  &  &  &  &  &  &  &  &  &  &  &  a_{14} & b_{14} & c_{14} & d_{14} &  &  &  &  &  &  &  & \\
  &  &  &  &  &  &  &  &  &  &  &  &  &  a_{15} & b_{15} & c_{15} & d_{15} &  &  &  &  &  &  & \\
  &  &  &  &  &  &  &  &  &  &  &  &  &  &  a_{16} & b_{16} & c_{16} & d_{16} &  &  &  &  &  & \\
  &  &  &  &  &  &  &  &  &  &  &  &  &  &  &  a_{17} & b_{17} & c_{17} & d_{17} &  &  &  &  & \\
  &  &  &  &  &  &  &  &  &  &  &  &  &  &  &  &  a_{18} & b_{18} & c_{18} & d_{18} &  &  &  & \\ \hline
  &  &  &  &  &  &  &  &  &  &  &  &  &  &  &  &  &  a_{19} & b_{19} & c_{19} & d_{19} &  &  & \\
  &  &  &  &  &  &  &  &  &  &  &  &  &  &  &  &  &  &  a_{20} & b_{20} & c_{20} & d_{20} &  & \\
  &  &  &  &  &  &  &  &  &  &  &  &  &  &  &  &  &  &  &  a_{21} & b_{21} & c_{21} & d_{21} & \\
  &  &  &  &  &  &  &  &  &  &  &  &  &  &  &  &  &  &  &  &  a_{22} & b_{22} & c_{22} & d_{22}\\
  &  &  &  &  &  &  &  &  &  &  &  &  &  &  &  &  &  &  &  &  &  a_{23} & b_{23} & c_{23}\\
  &  &  &  &  &  &  &  &  &  &  &  &  &  &  &  &  &  &  &  &  &  &  a_{24} & b_{24}
\end{matrix}\]

\hypertarget{header-n11}{%
\subsection{解法 :}\label{header-n11}}

\begin{Shaded}
\begin{Highlighting}[]
\NormalTok{     \#\#\# 第一步}
\end{Highlighting}
\end{Shaded}

首先对\(i=0,1,\cdots,p-1\),消去\(A_{(i,j)}\)中对角线以下元素,即

\[\large e_{m_i+1,m_i+\alpha-1,1:\alpha-1} \gets a_{m_i+1,m_i+\alpha-1,m_i-\alpha+2:m_i}\]

且对\(k=m_{i}+1,m_{i}+2,\cdots,m_{i+1}\),\(l=l_1,l_1+1,\cdots,k-1\),依次执行

\[a_{k,l} \gets a_{k,l}/a_{l,l}; r_k \gets r_k+a_{k,l}r_l;\\
a_{k,j} \gets a_{k,j}+a_{k,l}a_{l,j},j=l+1,\cdots,\min(l+\beta-1,n);\\
\begin{align} \tag{1}
e_{k,j} \gets
\begin{cases}
        a_{k,l}e_{l,j},\\
        e_{k,j}+a_{k,l}e_{l,j},
\end{cases}
\begin{matrix}
l = l_1 且 k \ge m_i + \alpha\\
l \ne l_1 或 k < m_i + \alpha
\end{matrix}
,j=1,2,\cdots,\alpha-1
\end{align}\]

其中\(l_1=\max(m_i+1,k-\alpha+1)\),且(1)式对\(i=0\)不需计算,

其次,对\(i=0,1,\cdots,p-1\)消去\(A^{(i,j)}\)中对角线以上元素,即

\[\large f_{m_{i+1}-\beta+2:m_{i+1},1,\beta-1} \gets a_{m_{i+1}-\beta+2:m_{i+1},m_{m+1}+1,m_{i+1}+\beta-1}\]

且对\(k=m_{i+1},\cdots,m_i+2,m_i+l,l=l_2,l_2-1,\cdots,k+1\),依次执行

\begin{equation} a_{k,l} \gets a_{k,l}/a_{l,l};r_k \gets r_k + a_{k,l}r_l;\tag{2} \end{equation}

\begin{equation} e_{k,j} \gets e_{k,j}+a_{k,l}e_{l,j},j=1,2,\cdots,\alpha-1;\tag{3}
\end{equation}

\begin{align}
\tag{4}
f_{k,j} \gets
\begin{cases}
a_{k,l}f_{l,j},\\
f_{k,j}+a_{k,l}f_{l,j},
\end{cases}
\begin{matrix}
l = l_2 且 k < m_{i+1} - \beta + 2 \\
l \ne l_2 或 k \ge m_{i+1} - \beta + 2
\end{matrix}, j=1,2,\cdots,\beta-1,
\end{align}

其中\(l_2=\min(k+\beta-1,m_{i,1})\),且(3)式对\(i=0\)不需计算,(4)式对\(i=p-1\)不需计算。

当\(n=24,\alpha=2,\beta=3,p=4\) 且 \(n_i=6\)
时,完成第一步后,得到的新系统矩阵如下图所示

\[\small\begin{matrix}
 b_1& & & & & & f_1& g_1& & & & & & & & & & & & & & & & \\
 & b_2& & & & & f_2& g_2& & & & & & & & & & & & & & & & \\
 & & b_3& & & & f_3& g_3& & & & & & & & & & & & & & & & \\
 & & & b_4& & & f_4& g_4& & & & & & & & & & & & & & & & \\
 & & & & b_5& & f_5& g_5& & & & & & & & & & & & & & & & \\
 & & & & & b_6& f_6& g_6& & & & & & & & & & & & & & & & \\ \hline
 & & & & & e_7& b_7& & & & & & f_7& g_7& & & & & & & & & & \\
 & & & & & e_8& & b_8& & & & & f_8& g_8& & & & & & & & & & \\
 & & & & & e_9& & & b_9& & & & f_9& g_9& & & & & & & & & & \\
 & & & & & e_{10}& & & & b_{10}& & & f_{10}& g_{10}& & & & & & & & & & \\
 & & & & & e_{11}& & & & & b_{11}& & f_{11}& g_{11}& & & & & & & & & & \\
 & & & & & e_{12}& & & & & & b_{12}& f_{12}& g_{12}& & & & & & & & & & \\ \hline
 & & & & & & & & & & & e_{13}& b_{13}& & & & & & f_{13}& g_{13}& & & & \\
 & & & & & & & & & & & e_{14}& & b_{14}& & & & & f_{14}& g_{14}& & & & \\
 & & & & & & & & & & & e_{15}& & & b_{15}& & & & f_{15}& g_{15}& & & & \\
 & & & & & & & & & & & e_{16}& & & & b_{16}& & & f_{16}& g_{16}& & & & \\
 & & & & & & & & & & & e_{17}& & & & & b_{17}& & f_{17}& g_{17}& & & & \\
 & & & & & & & & & & & e_{18}& & & & & & b_{18}& f_{18}& g_{18}& & & & \\ \hline
 & & & & & & & & & & & & & & & & & e_{19}& b_{19}& & & & & \\
 & & & & & & & & & & & & & & & & & e_{20}& & b_{20}& & & & \\
 & & & & & & & & & & & & & & & & & e_{21}& & & b_{21}& & & \\
 & & & & & & & & & & & & & & & & & e_{22}& & & & b_{22}& & \\
 & & & & & & & & & & & & & & & & & e_{23}& & & & & b_{23}& \\
 & & & & & & & & & & & & & & & & & e_{24}& & & & & & b_{24}\\
\end{matrix}\]

\hypertarget{header-n27}{%
\subsubsection{第二步}\label{header-n27}}

设 \(\gamma = \alpha+\beta-2\),求解由

\[E^{(i,2)}x^{(i-1,2)}+D^{(i,2)}x^{(i,2)}+F^{(i,2)}x^{(i+1,1)}=r^{(i,2)},i=0,1,\cdots,p-2,\]

\[E^{(i+1,1)}x^{(i,2)}+D^{(i+1,1)}x^{(i+1,1)}+F^{(i+1,1)}x^{(i+2,1)}=r^{(i+1,1)},i=0,1,\cdots,p-2,\]

组成的\((p-1)\gamma\) 阶线性方程组,其中

\[E^{(i,1)}=e_{m_i+1:m_i+\beta-1,1:\alpha-1}, \quad E^{(i,2)}=e_{m_i-\alpha+2:m_{i+1},1:\alpha-1}, \\
F^{(i,1)}=f_{m_i+1:m_i+\beta-1,1:\beta-1}, \quad F^{(i,2)}=f_{m_i-1-\alpha+2:m_{i+1},1:\beta-1},\]

\[D_{(i,1)}=diag(a_{m_i+1,m_i+1},\cdots,a_{m_{i+\beta-1},m_{i+\beta-1}}),
\\
D_{(i,2)}=diag(a_{m_{i-1}-\alpha+2,m_{i+1}-\alpha+2},\cdots,a_{m_i+1,m_{i+1}}),\]

\[x^{(i,1)}=[x_{m_i+1},x_{m_i+2},\cdots,x_{m_i+\beta-1}]^T \\
r^{(i,1)}=[r_{m_i+1},r_{m_i+2},\cdots,r_{m_i+\beta-1}]^T\]

可以将以上线性方程组写为

\[By=s\]

其中

\[B=
\begin{bmatrix}
F^{(0,2)} & D^{(0,2)} \\
D^{(1,1)} & E^{(1,1)} & F^{(1,1)} \\
                  & E^{(1,2)} & F^{(1,2)} & D^{(1,2)} \\
                  &               & D^{(2,1)} & E^{(2,1)} & F^{(2,1)} \\
                  &                       &                       &     \ddots    & \ddots      & \ddots \\
                  &                       &               &               & E^{(p-2,2)} & F^{(p-2,2)} & D^{(p-2,2)} \\
                  &                       &               &               &                     & D^{(p-1,1)} & E^{(p-1,1)}
\end{bmatrix}\]

\[y=[ (x^{(1,1)})^T , (x^{(0,2)})^T , (x^{(2,1)})^T , (x^{(1,2)})^T,\cdots,(x^{p-1,1})^T,(x^{(p-2,2)})^T]^T, \\
s=[ (r^{(1,1)})^T , (r^{(0,2)})^T , (r^{(2,1)})^T , (r^{(1,2)})^T,\cdots,(r^{p-1,1})^T,(r^{(p-2,2)})^T]^T,\]

\hypertarget{header-n40}{%
\subsubsection{第三步}\label{header-n40}}

利用得到的\(x_{m_i-\alpha+2}, \cdots, x_{m_i}, x_{m_i+1}, \cdots, x_{m_i+\beta-1}, i=1, 2, \cdots, p-1\)求出所有其他解的分量,即进行如下操作:

\[x_k \gets
(r_k - \sum_{j=1}^{\beta-1} f_{k,j} x_{n_1-\beta+1+j}) / a_k,
\quad k=1, 2, \cdots, n1 - \alpha + 1\]

\begin{align}
x_{m_i+k} \gets
(r_{m_i+k}-\sum^{\alpha-1}_{j=1}e_{m_i+k_{i,j}}x_{m_i-\alpha+1+j}-\sum^{\beta-1}_{j=1}f_{m_i+k\cdot j}x_{m_{i-1}-\beta+1+j})/a_{m_i+k},
\end{align}
\quad
\begin{matrix}
k=\beta,\cdots,n_i-\alpha+1 \\
i=1,2,\cdots,p-2 \quad
\end{matrix},

\[x_{m_{p-1}+k} \gets
(r_{m_{p-1}+k}-\sum^{\alpha-1}_{j=1}e_{m_{p-1}+k\cdot j}x_{m_{p-1}-\alpha+1+j})/a_{m_{p-1}+k},k=\beta,\beta+1,\cdots,n_{p-1
}.\]

显然,带状线性方程组求解的分裂法可以直接推广到块带状线性方程组,对于循环带状线性方程组与循环块带状线性方程组,也可以类似求解,其求解算法跟带状线性方程组与块带状线性方程组的求解并无本质区别。

\end{document}
